% This is a LaTeX thesis template for Monash University.
% to be used with Rmarkdown
% This template was produced by Rob Hyndman
% Updated: 30 December 2021

\documentclass{monashthesis}

%%%%%%%%%%%%%%%%%%%%%%%%%%%%%%%%%%%%%%%%%%%%%%%%%%%%%%%%%%%%%%%
% Add any LaTeX packages and other preamble here if required
%%%%%%%%%%%%%%%%%%%%%%%%%%%%%%%%%%%%%%%%%%%%%%%%%%%%%%%%%%%%%%%

\author{}
\title{}
\def\degreetitle{Graduate Diploma of Applied Statistics}
\degrees{B.Sc. (Hons), University of Queensland}
\def\affiliation{School of Science and Technology}

% Add subject and keywords below
\hypersetup{
     %pdfsubject={The Subject},
     %pdfkeywords={Some Keywords},
     pdfauthor={},
     pdftitle={},
     pdfproducer={Quarto with LaTeX}
}


\bibliography{thesisrefs.bib}

\graphicspath{{figures/}}

\begin{document}

\pagenumbering{roman}

\titlepage

{\setstretch{1.2}\sf\tighttoc\doublespacing}

\hypertarget{sec-intro}{%
\section{Introduction \& Background}\label{sec-intro}}

\hypertarget{modelling-infectious-epidemics}{%
\subsection{Modelling Infectious
Epidemics}\label{modelling-infectious-epidemics}}

``Epidemic models are used to inform decisions on disease prevention,
surveillance, control and treatment and can be applied to new
epidemics'' (\textbf{bjørnstad2020?})

\hypertarget{compartment-models}{%
\subsection{Compartment Models}\label{compartment-models}}

``Compartmental models are the most frequently used type of epidemic
model. In this class of models, individuals can be in a finite number of
discrete states. Some of these states are simply labels that specify the
various traits of individuals. Of these, some will be changing with
time, such as age class, and others will be fixed, such as sex or
species. Other states indicate the progress of an infection: for
example, an individual can upon becoming infected, typically first enter
a state of latency, then progress to a state of infectiousness, and then
lose infected status to progress to a recovered/immune state. With each
state one can associate the subpopulation of individuals who are in that
particular state at the given time (e.g.~a female in a latent state of
infection). Often the same symbol is used as a label for a state and to
denote the corresponding subpopulation size, either as a fraction or as
a number (e.g.~I or Y for individuals in an infectious state)'' -
(\textbf{diekmann2010?})

\hypertarget{the-sir-compartmental-model}{%
\subsubsection{The SIR compartmental
model}\label{the-sir-compartmental-model}}

The simplest comparmental model of infectious disease spread is the SIR
Compartmental Model, with three compartments: S - for individuals
\emph{susceptible} to the disease; I - for \emph{infected} individuals;
and R - for previously infected individuals who have \emph{recovered}
(or been otherwise \emph{removed} from that compartment). With the
simplifying assumption of a constant population size, \(N\), i.e.

\begin{equation}\protect\hypertarget{eq-SIR_N}{}{
N = S + I + R
}\label{eq-SIR_N}\end{equation}

Individuals move between compartments in a fixed set of ways, they may
either become infected (moving from S -\textgreater{} I), or recover
from infection (I -\textgreater{} R).

At each time unit an infected individual can come into contact with, on
average, \(k\frac{S}{N}\) Susceptible individuals. \(\pi\) is the
probability of infecting somebody on coming in contact, so
\(\beta = k\pi\) is the average rate at which an infected indivudual
will infect a susceptible. Infected individuals recover at the constant
rate, \(\gamma\), with \(1/\gamma\) the mean recovery time.

\begin{longtable}[]{@{}ll@{}}
\toprule\noalign{}
Parameter & Interpretation \\
\midrule\noalign{}
\endhead
\bottomrule\noalign{}
\endlastfoot
\(\beta\) & Transmission rate \\
\(\gamma\) & Recovery rate \\
\end{longtable}

Thus, an SIR model can be represented by the schematic

\hypertarget{sir-dynamics}{%
\subsection{SIR Dynamics}\label{sir-dynamics}}

\hypertarget{ode-representation}{%
\subsubsection{ODE representation}\label{ode-representation}}

Given the rate parameters defined above, the change in compartment
composition over time can be described by the system of differential
equations

\begin{equation}\protect\hypertarget{eq-SIR-ODE}{}{
\begin{aligned}
& \frac{d S}{d t}=-\beta I \frac{S}{N} \\
& \frac{d I}{d t}=\beta I \frac{S}{N}-\gamma I \\
& \frac{d R}{d t}=\gamma I
\end{aligned}
}\label{eq-SIR-ODE}\end{equation} Importantly, if the total population
size is known, given Equation~\ref{eq-SIR_N}, R = N - S + I and the
entire system can be described by two of the equations in
Equation~\ref{eq-SIR-ODE}.

EXAMPLE

\hypertarget{limitations-of-deterministic-ode}{%
\paragraph{Limitations of Deterministic
ODE}\label{limitations-of-deterministic-ode}}

While Equation~\ref{eq-SIR-ODE} provides a neat solution for the
expected behaviour of a epidemic, it fails to caputure the variability
inherent in a complex process like disease spread.

\hypertarget{stochastic-process-markov-chain-representation}{%
\subsubsection{Stochastic Process (Markov Chain)
Representation}\label{stochastic-process-markov-chain-representation}}

A Stochastic Process is a collection of random variables, \(X_t\).

\hypertarget{markov-chains}{%
\paragraph{Markov chains}\label{markov-chains}}

A Markov chain is a sequence of random variables \(X_0, X_1, \ldots\)
taking values in \(S\) with the property that

\begin{equation}\protect\hypertarget{eq-Markov-Property}{}{
\begin{aligned}
P\left(X_{n+1}\right. & \left.=j \mid X_0=x_0, \ldots, X_{n-1}=x_{n-1}, X_n=i\right) \\
& =P\left(X_{n+1}=j \mid X_n=i\right),
\end{aligned}
}\label{eq-Markov-Property}\end{equation}

for all \(x_0, \ldots, x_{n-1}, i, j \in S\), and \(n \geq 0\). That is,
the state at the next time step is determined only by the state at the
current time step.

\hypertarget{time-homogeneity}{%
\paragraph{Time Homogeneity}\label{time-homogeneity}}

While not a general property of Markov chains, all of those considered
in this work will have the additional property of \emph{time
homogeneity}:

\[
P\left(X_{n+1}=j \mid X_n=i\right)=P\left(X_1=j \mid X_0=i\right),
\] i.e.~the probabilities in Equation~\ref{eq-Markov-Property} do not
vary with \(t\)

\hypertarget{sir-dtmc}{%
\paragraph{SIR DTMC}\label{sir-dtmc}}

An SIR Compartment model an be described by a Markov chain of two
independent random variables S(t) and I(t) representing the number of
susceptible or individuals infected at time t respectively (as in the
ODE case, a third RV R(t) denoting the number of recovered individuals
is fully determined when the population size is known, and can be left
out from this characterisation).

There are two potential events (tbl-SIR\_Events) resulting in a change
of state i.e.~from \(X_{t}\) to \(X_{t+\delta t}\)

\begin{equation}\protect\hypertarget{eq-SIR_DTMC_Trans}{}{
p _ {(s +k, i +j),(s,i)}(\Delta t) =
\begin{cases}
{\frac { \beta i s} { N}} \Delta t, & {(k, j) = (-1, 1) } \\ 
{ \gamma i \Delta t}, & { (k, j) = (0, -1)  } \\ 
{ 1 - \bigg[ \frac { \beta i s } { N } +  \gamma i }\bigg] \Delta t, & { (k, j) = (0, 0)} \\
0 & {otherwise}
\end{cases}
}\label{eq-SIR_DTMC_Trans}\end{equation}

\hypertarget{transition-matrix}{%
\subparagraph{Transition Matrix}\label{transition-matrix}}

\hypertarget{sir-ctmc}{%
\paragraph{SIR CTMC}\label{sir-ctmc}}


\end{document}
